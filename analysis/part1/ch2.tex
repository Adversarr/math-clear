\chapter{函数的连续性}

本章重点:

\begin{enumerate}
    \item 函数的概念
    \item 函数的极限
    \item 连续函数
\end{enumerate}

\section{集合的映射}

\begin{definition}[映射]
    设 $A,B$ 是两个集合,若 $f$ 表示一个规律,使得对于 $A$ 中的任何一个元素 $x$ 都有 $B$ 中的唯一元素(记为$f(x)$)与之对应,则称 $f$ 是一个从 $A$ 到 $B$ 的映射。用 $f:A\rightarrow B$ 表示。其中,$A$ 叫做映射 $f$ 的\textbf{定义域},$f(x)\in B$ 是$x$ 在映射 $f$ 下的\textbf{像}。
\end{definition}

\maketip 一个映射需要有两个基本要素:

\begin{enumerate}
    \item 定义域 $A$;
    \item 规律 $f$,即规定了这种二元关系。
\end{enumerate}

\begin{definition}[函数的相等]
    设 $f:A\rightarrow B$ 且 $g:A\rightarrow B$。若 $\forall x\in A$ 有 $f(x) = g(x)$,则称 $f$ 和 $g$ 相等,记作 $f=g$。
\end{definition}

\begin{definition}[满射]
    设 $f:A\rightarrow B$。如果 $f(A) = B$ 则称 $f$ 是从 $A$ 到 $B$ 的满射。
\end{definition}

\begin{note}一般的,设$E\subseteq A$,$f(E)$ 是指 $\{f(x): x\in E\}$}

\begin{definition}[单射]
    设 $f:A\rightarrow B$。对于任意 $x,y\in A,x\ne y$ 有 $f(x) \ne f(y)$,则 $f$ 为单射。
\end{definition}

\begin{definition}[一一映射]
    设 $f:A\rightarrow B$。若 $f$ 既是单射,又是满射,则称 $f$ 为一一映射。
\end{definition}