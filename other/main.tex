\documentclass[color=plain, cn]{elegantbook}
\usepackage{amsmath}

\title{实变函数与泛函分析}
\author{Jerry Yang}
\date{\today}
\version{1.00}
\institute{Southeast University, Nanjing, China}

\definecolor{main}{RGB}{70,70,70}
\definecolor{second}{RGB}{115,45,2}
\definecolor{third}{RGB}{0,80,80}
\definecolor{structurecolor}{RGB}{0,0,0}

\begin{document}
\maketitle
\tableofcontents
\part{P1}
\chapter{c1}
\section{s1}
\subsection{ss1}

\begin{definition}{实数}{reals}
	所谓实数,就是一个有理柯西列。
\end{definition}


\begin{theorem}{阿基米德性质}{Archimedean-property}
	设 $a,\varepsilon \in \mathbb R$ 为正数,则存在正整数 $M > 0$ 使得 $a < M\varepsilon$。
\end{theorem}


\begin{proposition}{}{}
	设 $m$,$n$ 是自然数,若 $m$ 是正数,那么 $m+n$ 也是正数
\end{proposition}


\begin{proposition}{}{}
	设 $m$,$n$ 是自然数,若 $m$ 是正数,那么 $m+n$ 也是正数
\end{proposition}

\begin{example}
	例如,$1 = 0++$
\end{example}


\begin{note}
	这里用到了递归定义的理论。
\end{note}

\end{document}