\section{实数上的极限理论}

\begin{definition}{极限}{}
	设 $\{a_n\}\subseteq \mathbb R$,且 $a\in \mathbb R$。我们称 $\{a_n\}$ 收敛到 $a$,当所有的实数 $\varepsilon > 0$ 有一个 $N\in \mathbb N$ 使得:
	
$$
|a_n - a | < \varepsilon
$$
\end{definition}

\begin{theorem}{柯西收敛原理}{}
	设 $\{a_n\}$ 是\textbf{实数列},则下列命题等价:
	
	\begin{enumerate}
		\item $\{a_n\}$ 是收敛的;
		\item 对于任意实数 $\varepsilon > 0$,存在一个 $N\in \mathbb N$ 使得对于任何 $m,n\ge N$ 都有 $|a_m - a_n| < \varepsilon$
	\end{enumerate}
\end{theorem}

\begin{theorem}{维尔斯特拉斯-单调有界定理}{}
	设 $\{a_n\}$ 是实数列,若单调增有上界,或单调减有下界则收敛。
\end{theorem}

\begin{theorem}{康托-闭区间套原理}{}
	设 $\{I_n\}$ 是一系列闭区间,若 $I_1\sup I_2 \sup \cdots \supset I_n\supset\cdots$,且 $b_n - a_n \rightarrow 0~(n\rightarrow \infty)$ 则 $a = \lim_{n\rightarrow \infty } a_n = \lim _{n\rightarrow \infty} b_n$。
\end{theorem}

\begin{theorem}{确界存在定理}{}
	设 $A\subset R$ 是非空有界集,则其上下界存在且唯一。
\end{theorem}

\begin{theorem}{波尔萨诺-维尔斯特拉斯定理}{}
	有界数列必有收敛子列。
\end{theorem}

\begin{theorem}{海涅-有限覆盖定理}{}
	设 $\mathcal F$ 是一族开区间,$F$ 是满足 $F\subset \bigcup _{O\in \mathcal F} O$ 的有界紧集,则存在有限的开区间(覆盖)使得 $F\subset \bigcup _{n=1} ^m O_n$
\end{theorem}











