\section{自然数}

\noindent\textbf{要点:}
\begin{enumerate}
	\item 什么是自然数?
	\item 什么是加法和乘法?
\end{enumerate}

\subsection{自然数集的导出}

从两个最基础的概念开始解释自然数:

\begin{itemize}
	\item 数$0$
	\item 增长操作
\end{itemize}

其中增长操作有如下的定义:

$$
\begin{aligned}
	1&=0++\\
	2&=(0++)++\\
	&\cdots\cdots\\
	n&=(\cdots(n++)++\cdots)++
\end{aligned}
$$

我们定义的这个系统的\textbf{公理}如下:

\begin{axiom}
	$0$ 是自然数
\end{axiom}

\begin{axiom}
	自然数的增长也是自然数
\end{axiom}

\begin{axiom}
	$0$ 不是任何自然数的增长
\end{axiom}

\begin{axiom}
	不同的自然数的增长不同。
\end{axiom}

\begin{axiom}[归纳原理]\label{thm:Induct-Axiom}
	若$p(n)$是一个与自然数相关的性质。假设$p(0)$成立,且对于任意的自然数$n$而言,$p(n)$成立是$p(n+1)$成立的充分条件。那么$p(n)$对于所有的自然数都成立。
\end{axiom}

\maketip{
	\begin{enumerate}
		\item 第三条表明了不会出现如 $3\ne 0$ 的情况;
		\item 第四条表明了不会出现 $3\ne 2$ 这样的情况;
		\item 第五条保证了自然数的增长操作产生的是两两不相同的数,是从\textbf{有限}到\textbf{无限}的过程。
	\end{enumerate}
}

\subsection{自然数的加法}

为了定义自然数的加法和乘法,我们借助两个基础的概念:

\begin{enumerate}
	\item 自然数集$\mathbb N$
	\item 函数$f:\mathbb N \rightarrow \mathbb N$
\end{enumerate}

\begin{theorem}[递归定义]\label{thm:RecursiveDefinition}
	假设对于任意的自然数$n\in \mathbb N$,都存在一个函数 $f_n:\mathbb N\rightarrow \mathbb N$。设$c$是一个自然数,则对于任意的$n\in \mathbb N$,都存在唯一的$a_n\in \mathbb N$使得$a_0=c$且$a_{n++} = f_n(a_n)$。
\end{theorem}

\maketip{也就是说,我们可以通过一个初值和一个函数去\textbf{构造}一个无穷数列}

\begin{definition}[自然数的加法]
	设 $m,n\in \mathbb{N}$,首先定义出 $0+m:=m$。假设 $n+m$ 已经定义,可以定义$(n++)+m:=(n+m)++$。
\end{definition}

\maketip{这里我们的函数是$f(n)=n++,\quad n\in \mathbb N$,利用归纳原理和递归定义,可以得到$m+n$的定义。例如:}

\[
\begin{aligned}
	&1+n=n++,\\
	&2+n=(n++)++,\\
	&3+n=((n++)++)++.
\end{aligned}
\]

\begin{theorem}[封闭性]
	设 $m,n\in \mathbb N$ 则 $n+m\in \mathbb N$。
\end{theorem}

\subsection{自然数加法的性质}

下面我们关注加法的交换律:

\begin{lemma}\label{lem1}
	设 $m\in \mathbb N$ 则 $m+0=m$。
\end{lemma}

\maketip{直接用数学归纳法证明}

\begin{lemma}
	设 $m,n\in \mathbb N$ 则 $n+(m++)=(n+m)++$。
\end{lemma}

\maketip{固定$m$,利用引理\ref{lem1}的结论和加法的定义式。}

\begin{corollary}
	若 $m\in \mathbb N$ 则有 $m++=m+1=1+m$。
\end{corollary}

通过以上的三个结论,我们可以得到自然数加法的交换律:

\begin{theorem}[交换律]
	设 $m,n\in \mathbb N$ 则 $n+m=m+n$。
\end{theorem}

\begin{proof}
	即证明:$n+(m++)=(n++)+m$
	
	依然利用数学归纳法:设$m$为任意的固定的自然数,证明对所有的 $n\in \mathbb N$ 都成立上式即可。
\end{proof}

下面的定理证明不再有详细的证明:

\begin{theorem}[结合律]
	若 $a,b,c\in \mathbb N$ 则 $(a+b)+c = a+(b+c)$
\end{theorem}

\maketip{三个自然数数的归纳法,依然是固定一个,对另外的进行归纳。}

\begin{theorem}[消去律]
	设 $a,b,c\in \mathbb N$ 若 $a+b=a+c$ 则 $b=c$
\end{theorem}

至此,我们得到了自然数的概念。这使得我们可以在下一小节中,依此定义构筑自然数的''序''的概念。

最后,我们在这里给出正数的概念:

\begin{definition}[正整数]
	正数是指一个 $m\in \mathbb N$,且满足 $m$ 不等于 $0$。
\end{definition}

\begin{theorem}
	设 $m$,$n$ 是自然数,若 $m$ 是正数,那么 $m+n$ 也是正数
\end{theorem}

\maketip{设$a,b\in \mathbb N$,若 $a+b=0$,则 $a=0$ 且 $b=0$}

\subsection{自然数的序}

自然数的序关系可以通过以下符号表示:

$$
\ge,\quad =,\quad\le,\quad\le,\quad >,\quad <
$$

\begin{definition}[自然数的序]
	设 $m,n\in \mathbb N$。若存在 $a\in \mathbb N$,使得$n=m+a$,则称 $n$ 大于等于 $m$,记作 $n\le m$ 或 $m\le n$。若 $n\ge m$ 且 $n\ne m$ 则称 $n$ 大于 $m$ 记作 $n > m$ 或 $m < n$。
\end{definition}

\begin{theorem}[序的基本性质]
	设 $a,b,c\in \mathbb N$,下式成立:
	\begin{itemize}
		\item $a\ge a$;
		\item 若 $a\ge b$ 且 $b \ge c$,则$a\ge c$;
		\item 若 $a\ge b$ 且 $b \ge a$,则$a=b$;
		\item $a \ge b$ 当且仅当 $a+c\ge b+c$ 对于所有正整数或对于某个 $c\in \mathbb N$ 成立;
		\item $a < b$ 当且仅当存在一个正数 $d\in \mathbb N$ 使得 $b=a+d$ 或 $a++\le b$。
	\end{itemize}
\end{theorem}

\begin{theorem}[三奇性]
	若 $a,b\in \mathbb N$ 则下列结论成立且仅成立一个:
	
	$$
	a\ge b,\quad a=b, \quad a<b
	$$
\end{theorem}

\maketip{仅仅成立一个是容易证的,只需要使用反证法即可。但证明有一个成立是难的,需要固定其中一个($a$ 或 $b$),再使用归纳法证明。}

\subsection{自然数的乘法}

\begin{definition}
	设 $m,n \in \mathbb N$。定义 $0 \times m = 0$,假设 $n\times m$ 已经定义,则我们定义 $(n++)\times m = (n\times m) + m$
\end{definition}

\maketip{这里的定义与加法相同,用了递归定义\ref{thm:RecursiveDefinition}来进行定义。其中 $f_m(n) = n+m$,则$a_{n++} = f(a_n)$。}

利用归纳原理\ref{thm:Induct-Axiom},可以得到以下关于乘法的性质:

\begin{theorem}[乘法的封闭性]
	乘法对自然数集是封闭的,即若$m,n\in \mathbb N$,那么 $m\times n\in \mathbb N$
\end{theorem}

\begin{theorem}[交换律]
	乘法满足交换律,即若 $m,n\in \mathbb N$,那么$n\times m = m\times n$。
\end{theorem}

\maketip{下面,我们直接用 $ab$ 表示 $a\times b$}

\begin{theorem}[乘法的其他性质]
	设 $a,b\in \mathbb N$,那么下列结论成立:
	\begin{itemize}
		\item 若 $a,b$ 为正数,那么 $ab$ 也是正数;
		\item $ab=0$ 当且仅当$a=0$或$b=0$;
		\item $ab\ne 0$ 当且仅当 $a\ne 0$ 且 $b\ne 0$
	\end{itemize}
\end{theorem}

\begin{theorem}[分配律]
	若 $a,b,c\in \mathbb N$ 那么:
	$$
	a(b+c) = ab+ac
	$$
	\noindent 且
	$$
	(b+c)a = ba+ca
	$$
\end{theorem}

\begin{theorem}[结合律]
	若 $a,b,c\in\mathbb N$ 则
	$$
	(ab)\times c = a\times (bc)
	$$
\end{theorem}


\begin{theorem}[保序性]
	设 $a,b\in \mathbb N$,且 $a<b$。若 $c\in \mathbb N$ 为正数,则 $ac<bc$。
\end{theorem}

\begin{theorem}[消去律]
	设$a,b,c\in \mathbb N$,且 $ac=bc$。若 $c$ 为正数,则$a=b$。
\end{theorem}

\begin{theorem}[欧几里得算法]
	设 $n\in \mathbb N$,且 $q\in \mathbb N$为正数,则存在两个自然数,使得$0\le r < q$ 且 $n=mq+r$。
\end{theorem}

\maketip{欧几里得算法是\textbf{数论}的起始点。}

\begin{definition}[幂]
	设 $m\in \mathbb N$,则定义 $m^0=1$,递归定义出$m^{n++}=m^n\times m$
\end{definition}


至此,我们已经构建了完备的自然数系 $(\mathbb N, +, \times)$,我们可以沿着:

$$
\mathbb N\rightarrow \mathbb Z\rightarrow \mathbb Q\rightarrow \cdots
$$

\noindent 来定义更多的数。

\section{整数}

