\section{整数}

\subsection{整数的定义}

\begin{definition}{整数}{}
	设 $a,b\in \mathbb N$,整数是指有这样的形式 $a\smile b$。如果$a+d=b+c$,那么两个整数$a\smile b, c\smile d$是\textbf{相等}的。整数的全体被记为为$\mathbb Z$。
\end{definition}

\begin{definition}{加法}{}
	两个整数的和为 $(a\smile b) + (c\smile d) = (a+c)\smile (b+d)$。
\end{definition}

\begin{definition}{乘法}{}
	两个整数的乘积为 $(a\smile b) \times (c\smile d) = (ac+bd)\smile(ad+bc)$
\end{definition}

\begin{note}
	很显然,这里的 $\smile$ 就是我们看到的减法,但是我们在这里没有定义减法,故用这个符号代替。实际上我们只需要一个有序数对,也就是用有序自然数对来表示整数$(a,b)$。
\end{note}

\begin{note} 
	这个定义下的''相等''关系具有自反性、对称性、传递性。并且满足代入定理
\end{note}

$$
a\smile b = a'\smile b' \Rightarrow 
\begin{cases}
	(a\smile b) + (c\smile d) = (a'\smile b') + (c\smile d)\\
	(c\smile d) + (a\smile b) = (c'\smile b') + ac\smile b)
\end{cases}
$$

\begin{definition}{负数}{}
	设 $a\smile b$ 是整数,其相反数定义为 $(b\smile a)$,记作 $-(a\smile b)$
\end{definition}

\begin{corollary}{}{}
	设 $m,n\in \mathbb N$ 则 $(-n)m = n(-m) = -(nm)$。
\end{corollary}

\begin{lemma}{}{}
	设 $x\in Z$ 则下列三条结论成立且仅成立一个:
	\begin{enumerate}
		\item $x$ 是 0;
		\item $x$ 等于一个正自然数;
		\item $x$ 是负整数。
	\end{enumerate}
\end{lemma}

\begin{theorem}{整数的代数运算规则}{}
	设 $x,y,z\in \mathbb Z$,则:
	$$
	\begin{aligned}
		x+y & = y + x\\
		(x+y)+z & = x + (y+z)\\
		x + 0 & = 0 + x = x\\
		x + (-x) & = (-x) + x = 0\\
		xy & = yx\\
		(xy)z & = x(yz)\\
		x1 & =1x=x\\
		x(y+z) &=xy+xz\\
		(y+z)x & = yx + zx
	\end{aligned}
	$$
\end{theorem}

从而,我们可以定义减法。实际上,我们可以得出结论:每一个整数都是两个自然数的差。

\begin{definition}{减法}{}
	设 $x, y\in \mathbb Z$,则其差被定义为:$x-y= x+(-y)$
\end{definition}

\begin{proposition}{}{}
	设 $x,y\in \mathbb N$,则$x-y=x\smile y$。
\end{proposition}

\begin{proposition}{}{}
	设 $a\in \mathbb Z$,则$a + (-a) = 0$。
\end{proposition}

\begin{proposition}{}{}
	设 $a\in \mathbb Z$,则$a = 0 \iff (-a) = 0$。
\end{proposition}

\begin{proposition}{}{}
	设 $a,b\in \mathbb Z$,则$a=b\iff a-b = 0$。
\end{proposition}

\begin{proposition}{}{}
	设 $a,b\in \mathbb Z$,则$ab=0 \iff a = 0$ 或 $b=0$。
\end{proposition}

\begin{proposition}{}{}
	设 $a,b, c\in \mathbb Z$,则$ab-ac=a(b-c)$。
\end{proposition}

\begin{proposition}{}{}
	设 $a,b, c\in \mathbb Z$,则$ac=bc$时,若$c\ne 0$ 则 $a=b$。
\end{proposition}

\subsection{整数的序}

\begin{definition}{序}{}
	设 $a,b\in \mathbb Z$。当存在$n\in \mathbb N$,使得$a=b+n$,我们称$a$ 大于等于$b$,记作 $a \ge b$ 或 $b \le a$。当 $a \ne b$ 且 $a\ge b$ 时,我们称 $a$ 大于 $b$,记作$a>b$或$b<a$。
\end{definition}

\begin{theorem}{整数的序的性质}{}
	设 $a,b,c\in \mathbb Z$ 则:
	\begin{enumerate}
		\item $a> b\iff a-b$时一个正实数
		\item 若 $a>b$ 则 $a+c>b+c$;
		\item 若 $a>b$ 且 $c$ 是整数,则 $ac>bc$;
		\item 若 $a>b$ 则 $-a<-b$;
		\item 若 $a>b$ 且 $b>c$ 则 $a>c$
		\item $a>b,a=b,a<b$ 必成立且仅成立其一。
	\end{enumerate}
\end{theorem}











