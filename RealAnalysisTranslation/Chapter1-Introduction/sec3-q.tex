\section{有理数}

\subsection{有理数的定义}\label{subsec:rational-numbers}

\begin{definition}[有理数]
	设 $a,b\in \mathbb Z$ 且 $b\ne 0$,有理数是指有形如 $a//b$ 的表达式。我们说 $a//b$ 和 $c//d$ 是相等的,当$ad=cb$。有理数集表示为$\mathbb Q$ 
\end{definition}

\begin{definition}[有理数加法]
	\textbf{两个有理数的和}表示为 $a//b + c//d =(ad+cb) // bd$
\end{definition}

\begin{definition}[有理数乘法]
	两个有理数的积定义为 $(a//b) \times (c//d) = (ac)//(bd)$
\end{definition}

\begin{definition}[相反数]
	有理数的相反数为:$-(a//b) = (-a)//b$。
\end{definition}

\begin{theorem}
	设 $x = a//b \in \mathbb Q$ 则 $x=0\iff a=0$
\end{theorem}

\begin{definition}[倒数]
	设 $x=a//b\ne 0$ 是有理数,则$x$的倒数为$x^{-1} = b//a$。
\end{definition}

\maketip{有这些定义,我们可以定义加法、乘法的逆运算。}

\begin{definition}[减法]
	设 $x,y\in \mathbb Q$ 则 $x-y$ 定义为 $x-y=x+(-y)$。
\end{definition}

\begin{definition}[除法]
	设 $x,y\in \mathbb{Q}$,其商被定义为 $x/y = x\times y^{-1}$。
\end{definition}

\maketip{在这里我们能看到,$a//b$实际上就可以看作$a/b$,但我们不能预先这样定义。}

\begin{theorem}[有理数的代数运算法则]
	设 $x,y,z\in \mathbb Q$,则:
$$
\begin{aligned}
	x+y&=y+x\\
	(x+y)+z & = x+(y+z)\\
	x+0 & = 0+x=x\\
	x + (-x)& = (-x) + x = 0\\
	xy&=yx\\
	(xy)z & = x(yz)\\
	x1&=1x=x\\
	x(y+z) & = xy+xz\\
	(y+z)x & = yx+zx\\
	x\ne 0 \Rightarrow xx^{-1} &= x^{-1} x=1
\end{aligned}
$$

\end{theorem}

\begin{definition}[正、负]
	如果$a$ 和 $b$都是正的,则有理数 $x=a//b$ 是正的。如果存在正有理数 $x=-y$ 则称 $y$ 是负的。
\end{definition}

\begin{lemma}
	设 $x,y\in \mathbb Q$,则
	\begin{enumerate}
		\item $x+(-x) = 0$;
		\item $x=y\iff -x=-y$;
		\item $x$ 是正的,当且仅当 $-x$ 是负的。
	\end{enumerate}
\end{lemma}

\begin{theorem}[有理数的三奇性]
	设 $x\in \mathbb Q$ 则下列结论成立且仅成立其一:
	
	\begin{enumerate}
		\item $x=0$;
		\item $x$ 是正有理数;
		\item $x$ 是负有理数。
	\end{enumerate}
\end{theorem}

\subsection{有理数的序}

\begin{definition}[有理数的序]
	设 $x,y\in \mathbb Q$,则当 $x-y$ 为正时,我们称 $x$ \textbf{大于} $y$,即$x>y$或$y<x$;当 $y-x$ 为正时,我们称 $x$ \textbf{小于} $y$,即$x<y$ 或 $y>x$。
	
	\noindent 另外,$x=y$或$x>y$ 记作 $x\ge y$,$x=y$或$x<y$ 记作 $x\le y$。
\end{definition}

\begin{theorem}[序的基本性质]
	设 $x,y,z\in \mathbb Q$,则下列结论成立:
	\begin{enumerate}
		\item 三条结论成立且仅成立其一:$x> y, \quad x=y,\quad x < y$;
		\item $x>y\iff y < x$
		\item $x>y \Rightarrow x+z > y+z$
		\item 若 $x > y$ 且 $z$ 是正数,则 $xz > yz$
		\item 若 $x > y$ 且 $z$ 是负数,则 $xz < yz$
	\end{enumerate}
\end{theorem}

至此,我们定义了有理数系统 $(\mathbb Q, +, -, \times , \div, \le)$。可以证明的是,该系统是一个全序域。最后我们将定义其中的''距离'',以构筑''实数''。

\subsection{绝对值、有理数的距离}

\begin{definition}
	设 $x\in \mathbb Q$。则 $x$ 的绝对值(记作$|x|$)定义为:
	\begin{enumerate}
		\item 若 $x=0$,则 $|x|=0$
		\item 若 $x > 0$,则 $|x|=x$
		\item 若 $x < 0$,则 $|x|=-x$
	\end{enumerate}
\end{definition}

\begin{corollary}
	对于任意 $x\in \mathbb Q$,有$\pm x \le |x|$。
\end{corollary}

\begin{theorem}[绝对值的基本性质]
	设 $x,y,z\in \mathbb Q$,下列结论成立:
	\begin{enumerate}
		\item (\textbf{非负性}) $|x| \ge 0$,且$|x|=0\iff x=0$;
		\item (\textbf{三角不等式}) $|x+y|\le |x| + |y|$;
		\item $-y \le x \le y \iff |x| \le y$;
		\item $|xy| = |x| |y|$,特别的$|-x| = |x|$。
	\end{enumerate}
\end{theorem}

\begin{definition}[距离]
	设 $x,y\in \mathbb Q$,则 $x$ 和 $y$ 之间的距离定义为 $|x-y|$,写作 $d(x,y)=|x-y|$。
\end{definition}

\begin{theorem}[距离的基本性质]
	设 $x,y,z\in \mathbb Q$,下列结论成立:
	\begin{enumerate}
		\item (\textbf{非负性}) $d(x,y)\ge 0$ 且 $d(x,y)=0\iff x=y$;
		\item (\textbf{对称性}) $d(x,y) = d(y,x)$;
		\item (\textbf{三角不等式}) $d(x,z) \le d(x,y) + d(y,z)$
	\end{enumerate}
\end{theorem}

\begin{theorem}
	设 $x\in Q$,则存在唯一的$n\in Z$ 使得 $n\le x < n+1$。
\end{theorem}

\begin{proof} 记$x = a//b,\quad ab\in\mathbb Z,b>0$,先证明存在性:
\begin{enumerate}
	\item $a\ge 0$ 用欧几里得算法,可得到两个数 $a=nb+r$,则
	
	$$
	nb \le a < (n+1b)\implies n\le x < n+1
	$$
	\item $a < 0$ 则 $-x > 0$,反过来用上式。
\end{enumerate}	

\noindent 唯一性:若存在两个$m,n\in \mathbb Z$,满足

$$
m\le x < m+1, \quad n\le x < n+1
$$

\noindent 则

$$
m \le x < n+1, \quad n \le x < m+1
$$

\noindent 消去 $x$,则:

$$
-1<m-n<1\implies m-n=0
$$
\end{proof}

\begin{theorem}
	设 $x,y\in \mathbb Q$ 且 $x < y$,则存在 $z\in \mathbb Q$,使得 $x<z<y$。
\end{theorem}

\maketip{直接构造$z=(x+y)/2$}

\maketip{也表明了两个有理数之间必然存在\textbf{无穷多个}有理数。}



















