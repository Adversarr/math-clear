\section{实数}

\subsection{实数定义}

\begin{definition}{有理数列}{}
	设 $m\in \mathbb Z$ 有理数列$\{a_n\}_{n=m}^\infty$是指任意的函数,从 $X=\{n\in \mathbb Z:n\ge m\}$ 映射到 $\mathbb Q$。更详细的说,其可以写作一列有理数:
	
	$$
	a_m,a_{m+1},a_{m+2},\cdots
	$$
\end{definition}

\begin{note}
    特别的,如果 $a_n=c\in \mathbb Q,\forall n\in X$ 则我们记作:

$$
\{c\}_{n=m}^\infty = \{a_n\}_{n=m}^\infty
$$

\end{note}

\begin{definition}{有理柯西列}{}
	设有理数列 $\{a_n\}_{n=1}^\infty$,其被称为柯西列,当对于任意有理数$\varepsilon>0$存在正整数 $N$ 使得对于任意$m,n\ge N$,有$|a_m-a_n|< \varepsilon$。
\end{definition}

\begin{definition}{有理柯西列的等价性}{}
	设 $\{a_n\}_{n=1}^\infty$ 和 $\{b_n\}_{n=1}^\infty$ 都是柯西列,若 $\forall \varepsilon \in Q,\varepsilon > 0$,都存在自然数 $N$,使得对于任意 $n\ge N$ 有 $|a_n-b_n|< \varepsilon$,则称 $\{a_n\}_{n=1}^\infty$ 和 $\{a_n\}_{n=1}^\infty$ 是等价的。
\end{definition}

\begin{lemma}{有理柯西列的子列}{}
	有理柯西列的所有子列都是有理柯西列,且与原列等价。
\end{lemma}

\begin{lemma}{有界性}{}
	有理柯西列是有界的。
\end{lemma}

\begin{lemma}{加法、乘法封闭性}{}
	两个有理柯西列的和/积是有理柯西列。
\end{lemma}

\begin{lemma}{}{}
	设 $\{a_n\}^\infty_{n=1},\{a_n'\}^\infty_{n=1},\{b_n\}^\infty_{n=1},\{b_n'\}^\infty_{n=1}$ 是四个柯西序列,
\end{lemma}

\begin{definition}{实数}{}
	\textbf{{\heiti 实数是有理柯西列}};实数的相等,定义为有理柯西列的等价。
\end{definition}

\begin{definition}{实数的加法}{}
	设 $a=\{a_n\}^\infty_{n=1}$ 和 $b=\{b_n\}^\infty_{n=1}$ 是两个实数,则其加法定义为:
	
	$$
	a+b = \{a_n+b_n\}^\infty_{n=1}
	$$
\end{definition}

\begin{definition}{实数的乘法}{}
	设 $a=\{a_n\}^\infty_{n=1}$ 和 $b=\{b_n\}^\infty_{n=1}$ 是两个实数,则其乘法定义为:
	
	$$
	ab = \{a_n b_n\}^\infty_{n=1}
	$$
\end{definition}

\begin{definition}{实数的相反数}{}
	设 $a = \{a_n\}^\infty_{n=1}$ 是实数,则其相反数(记为$-a$)为
	
	$$
	-a= \{-1\}^\infty_{n=1} \times \{a_n\}^\infty_{n=1} = \{-a_n\}^\infty_{n=1}
	$$
\end{definition}

类似于有理数\ref{subsec:rational-numbers},定义减法:

\begin{definition}{实数的减法}{}
	设 $a=\{a_n\}^\infty_{n=1}$ 和 $b=\{b_n\}^\infty_{n=1}$ 是两个实数,则其加法定义为:
	
	$$
	a-b = a + (-b)
	$$
\end{definition}

\subsection{非零实数及其相反数}

\begin{example} 观察实数 $0$:

$$
0 = \{1/n\}_{n=1}^\infty = \{1/2n\}_{n=1}^\infty
$$
\end{example}

\begin{theorem}{实数表示的不唯一性}{}
	设 $a=\{a_n\}_{n=1}^\infty$ 是非零实数,则存在柯西列满足:
	\begin{enumerate}
		\item 存在 $c\in \mathbb Q$,使得$|b_n|\ge c$
		\item $a=\{b_n\}^\infty_{n=1}$
	\end{enumerate}
\end{theorem}

\begin{definition}{非退化柯西列}{}
	设 $\{a_n\}^\infty_{n=1}$。当存在有理数 $c > 0$,使得对于任意$n\in \mathbb N$,有$|a_n| \ge c$。
\end{definition}

\begin{note}
	这表明了\textbf{对于任意一个非零实数,都可以表示为一个非退化的柯西列}。
\end{note}

\begin{definition}{相反数}{}
	设 $a = \{a_n\}^\infty_{n=1}$ 非 $0$,则其相反数被定义为:
	
	$$
	a^{-1} = \{a_n^{-1}\} _{n=1}^\infty
	$$
\end{definition}

\subsection{实数系统}

\begin{note}
	实数系统是指 $(\mathbb R, +, -, \times, \div)$
\end{note}

\begin{definition}{减法}{}
	设 $a,b\in \mathbb R$ 则 $a$ 和 $b$ 的减法定义为 $a-b = a+ (-b)$。
\end{definition}

\begin{definition}{实数的商}{}
	设 $a,b\in \mathbb R$ 且 $b \ne 0$ 则 $a/b$ 定义为 $a/b = a \times b ^{-1}$
\end{definition}

\begin{theorem}{代数运算规则}{}
	设 $x,y,z\in \mathbb R$,则:
$$
\begin{aligned}
	x+y&=y+x\\
	(x+y)+z & = x+(y+z)\\
	x+0 & = 0+x=x\\
	x + (-x)& = (-x) + x = 0\\
	xy&=yx\\
	(xy)z & = x(yz)\\
	x1&=1x=x\\
	x(y+z) & = xy+xz\\
	(y+z)x & = yx+zx
\end{aligned}
$$
\end{theorem}

\subsection{绝对值、序、距离}

\begin{definition}{正数、负数}{}
	实数 $a$ 是正的,如果其对应的柯西列满足对于某个有理数 $c$ 由 $a_n \ge c, n\in \mathbb N$。反之,若其可以写作 $a=-b$ 且 $b$ 为正数,那么称 $a$ 是负的。
\end{definition}

和有理数相同,我们可以得到实数的三奇性:

\begin{proposition}{}{}
	设 $a\in \mathbb R$ 则下列三条结论成立且仅成立一条:
	
	\begin{enumerate}
		\item $a$ 是 0
		\item $a$ 是正的
		\item $a$ 是负的
	\end{enumerate}
\end{proposition}

\begin{note}
	可以直接用定义证明。
\end{note}

\begin{proposition}{}{}
	设 $a\in \mathbb R$ 则下列命题是等价的:
	
	\begin{enumerate}
		\item $a > 0$
		\item 存在有理数 $c > 0$ 和柯西列 $\{a_n\}= a$,使存在某个$N \in \mathbb N, \forall n \ge N,a_n \ge c$
		\item 存在有理数 $d > 0$ 使得对于任意 $\{a_n\} = a$,存在 $N'\in \mathbb N$ 满足 $\forall n\in \mathbb N,a_n \ge d$
	\end{enumerate}
\end{proposition}

\begin{theorem}{实数的基本性质}{}
	设 $a,b\in \mathbb R$ 下列命题成立:
	
	\begin{enumerate}
		\item $a$ 是负的,当且仅当 $-a$ 是正的;
		\item 如果 $a,b$ 都是正数,则 $a+b$ 和 $a\times b$ 都是正数
		\item 如果 $a,b$ 都是负数,则 $a+b$ 是负数,$a\times b$ 是正数
	\end{enumerate}
\end{theorem}

\begin{definition}{绝对值}{}
	设 $a\in \mathbb R$。则 $a$ 的绝对值(记作$|a|$)定义为:
	\begin{enumerate}
		\item 若 $a=0$,则 $|a|=0$
		\item 若 $a > 0$,则 $|a|=a$
		\item 若 $a < 0$,则 $|a|=-a$
	\end{enumerate}
\end{definition}

\begin{definition}{实数的距离}{}
	设 $a,b\in \mathbb R$ 则其距离为 $d(a,b) = |a-b|$。
\end{definition}

\begin{definition}{实数的序}{}
	设 $a,b\in \mathbb R$,我们称 $a$ 大于 $b$(记为$a>b$),当 $a-b$ 是正数。反之,我们称 $a$ 小于 $b$(记作$a<b$),当 $a-b$ 是负数。另外的,我们定义 $a\ge b$ 当 $a > b$ 或 $a=b$;$a\le b$ 当 $a < b$ 或 $a=b$。
\end{definition}

\begin{theorem}{实数序的性质}{}
	设 $a,b\in \mathbb R$ 则下列命题等价:
	\begin{enumerate}
		\item $a>b$
		\item 存在有理数 $c > 0$,和两个柯西列,使得 $\{a_n\} = a$ 和 $\{b_n\} = b$ 且由 $a_n - b_n \ge c$
		\item 存在有理数 $d > 0$ 使得对于任何两个柯西列 $\{a'_n\} = a$ 和 $\{b_n'\} = b$,存在一个 $N'\in \mathbb N$ 使得 $a_n-b_n \ge d(n\ge N')$
	\end{enumerate}
\end{theorem}

\begin{theorem}{实数序的性质}{}
	设 $a,b, c \in \mathbb R$ 则下列命题成立:
	
	\begin{enumerate}
		\item $a<b,~a=b,~a<b$ 三者成立其一;
		\item 若 $a>b$ 且 $b>c$ 则 $a>c$
		\item 若 $a > b$ 且 $c$ 是正数,则 $ac > bc$
	\end{enumerate}
\end{theorem}

\begin{corollary}{}{}
	设 $a,b\in \mathbb R$ 是正实数,则下列命题成立:
	
	\begin{enumerate}
		\item $a^{-1}$ 是正数
		\item 若 $a> b$ 则 $a^{-1} < b ^{-1}$
	\end{enumerate}
\end{corollary}

\begin{theorem}{绝对值的基本性质}{}
	设 $a,b,c\in \mathbb R$ 则下列命题成立:
	
	\begin{enumerate}
		\item (\textbf{非退化性}) $0\le |a|$ 且 $|a| = 0 \iff a = 0$
		\item (\textbf{三角不等式}) $|a+b| \le |a| + |b|$
		\item $-b \le a \le b \iff |a| \le b$
		\item $|ab| = |a| |b|$。
	\end{enumerate}
\end{theorem}

\begin{theorem}{距离的基本性质}{}
	设 $a,b,c\in \mathbb R$ 则下列命题成立:
	
	\begin{enumerate}
		\item (\textbf{非退化性}) $0\le d(a,b)$ 且 $d(a,b) = 0 \iff a = b$
		\item (\textbf{对称性}) $d(a,b) = d(b, a)$
		\item (\textbf{三角不等式}) $d(a,c) \le d(a,b) + d(b,c)$
	\end{enumerate}
\end{theorem}

\subsection{阿基米德性质}

\begin{proposition}{}{}
	设 $a = \{a_n\}$ 是非负有理柯西列,则 $a$ 是非负实数。
\end{proposition}

\begin{corollary}{}{}
	设 $a = \{a_n\}$,$b = \{b_n\}$ 是两个柯西列。若 $a_n\ge b_n,\forall n$ 则 $a \ge b$
\end{corollary}

\begin{proposition}{}{}
	设 $a\in \mathbb R$ 是正数,则存在两个数 $q\in \mathbb Q, N\in \mathbb Z$ 使得 
	
	$$
	0 < q \le a \le N.
	$$
\end{proposition}

\begin{theorem}{阿基米德性质}{}
	设 $a,\varepsilon \in \mathbb R$ 为正数,则存在正整数 $M > 0$ 使得 $a < M\varepsilon$。
\end{theorem}

\maketip{表明了不存在\textbf{最大}数。}

\begin{proposition}{}{}
	设 $a,b\in \mathbb R$,且 $a<b$。则存在 $q\in \mathbb Q$ 使得 $a<q<b$。
\end{proposition}














